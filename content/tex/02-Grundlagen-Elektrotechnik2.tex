%ju 12-Jun-22 02-Grundlagen-Elektrotechnik2.tex
\section{Pulsweitenmodulation}\label{pulsweitenmodulation}

\begin{enumerate}
\item
  \textbf{PWM-Signal:} Rechtecksignal (vgl. Kennlinie)
\item
  \textbf{Messen:} Oszilloskop
\item
  \textbf{Ansteuerung:} $0 - 100~\%$, Impuls-Pause-Verhältnisänderung
  (nur Pulsweite ändert sich), aber Periodendauer / Frequenz bleibt
  konstant (keine Spannungsänderung!)
\item
  \textbf{Anwendung}

  \begin{itemize}
  \item
    Elektromotor (Drehzahl)
  \item
    Glühlampen (Helligkeit, dimmen)
  \item
    Beleuchtung, Beispiel: Einfaden-Glühlampe (Standlicht + Bremslicht)
  \end{itemize}
\end{enumerate}

Die \textbf{Pulsweitenmodulation} (PWM) ist eine digitale
Modulationsart, bei der eine technische Größe zwischen zwei Werten
(Spannungspegel $0~V$ und VCC) wechselt. Dabei wird bei konstanter
Frequenz ein Rechtecksignal moduliert, das in Pulsweite (\emph{oder}
Breite bzw. Länge) variiert. Das Verhältnis zwischen Impuls und Pause
(Ein- und Ausschaltzeit) wird als \textbf{Tastgrad / Tastverhältnis}
bezeichnet. \textbf{Anwendung} in der Steuer- und Regelungstechnik.

\section{Halbleiter}\label{halbleiter}

\textbf{Halbleiter} der Widerstand ist temperaturabhängig.

\textbf{Material} Beispiel: reines Silizium, wenn es erwärmt wird, wird
es leitend.

\textbf{Dotieren} Leitfähigkeit verbessern.

\subsection{P-Leiter und N-Leiter}\label{p-leiter-und-n-leiter}

\begin{enumerate}
\item
  \textbf{P-Leiter} \emph{Elektronenmangel} (Leitermaterial Silizium
  wird verunreinigt mit Bohr)
\item
  \textbf{N-Leiter} \emph{Elektronenüberschuss} (Leitermaterial Silizium
  wird verunreinigt mit Phosphor)
\end{enumerate}

$\to$ Spannung anlegen, dann kommt es zum Elektronenfluss, weil die
Elektronen versuchen sich auszugleichen.

\subsection{Sensoren und Aktoren}\label{sensoren-und-aktoren}

\textbf{Sensoren} erfassen Messgrößen und wandeln diese in elektrische
Signale um. (Sinne des Autos)

\textbf{Aktoren} Ausführen

\textbf{NTC und PTC}

\begin{enumerate}
\item
  \textbf{Heißleiter} $\to$ NTC-Widerstände

  \begin{itemize}
  \item
    \emph{negativer Temperatur-Koeffizient} $\to$ fallender Widerstand
    bei steigender Temperatur $\uparrow \downarrow$
  \item
    vgl. Kennlinie (x = Temperatur, y = Widerstand)
  \item
    \textbf{Anwendung:} Temperatursensoren (Kühlmittel-, Kraftstoff-,
    Luft-, Motortemperaturfühler)
  \item
    Schaltsymbol
  \item
    typische Werte: Kalt: $2 - 4~k\Omega$ und Warm:
    $200 - 400~\Omega$
  \item
    \emph{Leitfähigkeit} steigt mit zunehmender Temperatur
  \item
    Widerstand und Spannungsfall am NTC wierden kleiner
  \end{itemize}
\item
  \textbf{Kaltleiter} $\to$ PTC-Widerstände

  \begin{itemize}
  \item
    \emph{positiver Temperatur-Koeffizient} $\to$ zunehmender
    Widerstand bei steigender Temperatur $\uparrow \uparrow$
  \item
    vgl. Kennlinie (x = Temperatur, y = Widerstand)
  \item
    \textbf{Anwendung:} Hochtemperaturbereich
    (Diesel-Abgastemperatursensoren), Glühkerzen, Heizungen von
    Lambdasonden
  \item
    Schaltsymbol
  \item
    \emph{Leitfähigkeit} verringert sich mit zunehmender Temperatur
  \item
    Widerstand und Spannungsfall am PTC werden größer
  \item
    >>Je höher der Widerstand, desto geringer der Stromfluss.<<
  \end{itemize}
\end{enumerate}
