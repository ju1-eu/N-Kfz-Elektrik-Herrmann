%ju 06-Jun-22 02-Grundlagen-Elektrotechnik2.tex
\section{Pulsweitenmodulation}\label{pulsweitenmodulation}

\begin{enumerate}
\item
  \textbf{PWM-Signal:} Rechtecksignal
\item
  \textbf{Messen:} Oszilloskop
\item
  \textbf{Ansteuerung:} $0 - 100~\%$, Impuls - Pause -
  Verhältnisänderung / Frequenzänderung (keine Spannungsänderung!)
\item
  \textbf{Anwendung}

  \begin{itemize}
  \item
    Motorgeschwindigkeit
  \item
    Glühlampenhelligkeit
  \item
    Beleuchtung, Beispiel: Einfaden-Glühlampe (Standlicht + Bremslicht)
  \end{itemize}
\end{enumerate}

Die \textbf{Pulsweitenmodulation} (PWM) ist eine digitale
Modulationsart, bei der eine technische Größe zwischen zwei Werten
wechselt. Dabei wird bei konstanter Frequenz ein Rechtecksignal
moduliert, das in Weite, Breite bzw. Länge variiert. Das Verhältnis
zwischen Impuls und Pause wird als \textbf{Tastgrad} bezeichnet.
\textbf{Anwendung} in der Steuer- und Regelungstechnik.

\section{Halbleiter}\label{halbleiter}

\textbf{Material:} reines Silizium, wenn es wärmer wird, wird es
leidend.

\textbf{Dotieren}, um die Leitfähigkeit zu verbessern.

\begin{enumerate}
\item
  \textbf{P-Leiter} Silizium wird verunreinigt, mit Bohr, weniger
  Elektronen
\item
  \textbf{N-Leiter} Silizium wird verunreinigt, mit Phosphor,
  Elektronenüberschuss
\end{enumerate}

$\to$ Spannung anlegen, dann kommt es zum Elektronenfluss, weil die
Elektronen versuchen sich auszugleichen.

\textbf{Anwendung:} Diesel-Abgastemperatursensoren (PTC-Widerstände)

\section{Fragen zum Schaltplan Audi A3
2009}\label{fragen-zum-schaltplan-audi-a3-2009}

\textbf{1. Beschreiben Sie die Spannungsversorgung für die Pumpe für
Kühler der Abgasrückführung.}

\begin{enumerate}
\item
  Stromverteiler (87)
\item
  über Sicherung (24)
\item
  Steckverbindung ($T_{40/11}$)
\item
  Plusverbindung $B_{321}$
\item
  Pumpe für Kühler der Abgasrückführung $V_{400}$ (Pin 2)
\item
  Steckverbindung ($T_{14/9}$)
\item
  Masseverbindung (394)
\item
  Massepunkt (655) am Scheinwerfer links
\end{enumerate}

\textbf{2. Welche Fehler können auftreten und welche Auswirkungen haben
diese?}

\textbf{Welches Signal erwarten wir?} PWM-Signal

\begin{enumerate}
\item
  Bauteil defekt
\item
  Sicherung defekt
\item
  Leitungsunterbrechung (Signalleitung, Plusleitung, Masseleitung)
\item
  Übergangswiderstand (Spannungsfall im Stecker oder Leitungen)
\item
  Masseschluss oder Plusschluss
\end{enumerate}

\textbf{Mögliche Fehler}

\begin{enumerate}
\item
  MIL-Lampe an $\to$ Abgas relevanter Fehler (Steuerhinterziehung)
\item
  AGR klemmt fest, wenn offen $\to$ Leistungsverlust im warmen Zustand
\end{enumerate}

\textbf{3. Fehler Abgastemperaturgeber 1, 3, 4 sind im Fehlerspeicher
abgelegt. Nennen Sie mögliche Fehlerursachen.}

\begin{enumerate}
\item
  Bauteil defekt (Temperatursensoren)
\item
  plusseitig
\end{enumerate}

\textbf{4. Motordrehzahlgeber hat kein Signal. Welche möglichen Ursachen
liegen vor?}

\textbf{Was erwarten wir?} Hallgeber, Versorgungsspannung (5 V)

\begin{enumerate}
\item
  Masseverbindung
\item
  Bauteil defekt
\item
  Plusverbindung (5 V)
\end{enumerate}

\textbf{5. Ihnen liegt ein Fehler zur Kraftstoffvorratsanzeige vor.
Nennen Sie mögliche Fehler.}

\begin{enumerate}
\item
  Geber defekt
\item
  Gebermasse
\item
  Übergangswiderstand
\item
  Leitungsunterbrechung von (Pin 3 + 4) zum Kombiinstrument
\item
  Widerstand defekt
\end{enumerate}
